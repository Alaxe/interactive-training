\documentclass[a4paper,11pt]{article}

\usepackage[a4paper, left=2.5cm, right=2.5cm, bottom=2.5cm, top=3.5cm,
headheight=1cm]{geometry}
\usepackage{polyglossia}
\usepackage{amsmath}
\usepackage{fancyhdr}
\usepackage{titlesec}
\usepackage{enumitem}


\setmainfont[Ligatures=TeX]{Noto Serif}
\newfontfamily\cyrillicfont[Script=Cyrillic]{Noto Serif}

\newcommand{\code}[1]{\texttt{#1}}
\newcommand{\noskip}{\vspace{-\parskip}}

\setmainlanguage{bulgarian}
\PolyglossiaSetup{bulgarian}{indentfirst=true}

\setlength{\parindent}{0em}
\setlength{\parskip}{1em}
\titlespacing{\section}{0pt}{0pt}{0pt}
\setlist{nosep}

\renewcommand{\labelitemi}{$\bullet$}
\renewcommand{\labelitemii}{$\circ$}
\renewcommand{\thesection}{}

\pagestyle{fancy}
\fancyhf{}
\rfoot{\thepage}
\chead{\uppercase{Национална школа по информатика \\Стара Загора, 24-30 август
2017 г.}}
\renewcommand{\headrulewidth}{0pt}

\title{Фалшиви монети}
\author{Александър Кръстев}
\date{}

\makeatletter
\begin{document}
{%
    \centering \LARGE 
    \textbf{\@title}
    \par
}
%\vspace{0.5cm}

Ави и Боби откриват съкровище от $N=2^k$ монети (т.е. $N$ е степен на $2$),
номерирани $0, 1, ..., (N-1)$.
Те знаят, че имат $N-2$ монети, тежащи $10$г, $1$ монета, тежаща $9$г
и една монета, тежаща $11$г.
Децата искат да разберат коя е $11$ грамовата монета, защото смятат, че
за разлика от останалите тя е направена от чисто злато.
Ави и Боби разполагат единствено с везна (без теглилки) и ограничено време,
за което могат да направят най-много $Q$ претегляния.

Вашата задача е да напишете програма, която да помогне на Ави и Боби да открият
по-тежката монета.

\section*{Детайли по имплментацията}
Вашата програма трябва да съдържа в началото си \code{\#include "fakecoins.h"}.
\\
Вашата програма трябва да реализира следната функция:

\noskip
\begin{itemize}
    \item \code{int find\_heavier(int N)}
    \begin{itemize}%[noitemsep,topsep=0pt] 
        \item \code{N}: броят на манетите на Ави и Боби. Гарантирано е, че 
            $N = 2^k$.
        \item Функцията трябва да връща на индекса на по-тежката монета 
            (монетите са индексирани от $0$).
        \item Функцията се вика веднъж по време на изпълнението на програмата.
    \end{itemize}
\end{itemize}

Вашата програма може да задава върпоси, използвайки следната функция:
\noskip
\begin{itemize}
    \item \code{int compare\_coins(std::vector<int> a, std::vector<int> b)}
    \begin{itemize}
        \item \code{a}: индексите $a_0, a_1, ..., a_{M-1}$, на монетите, които
            Ави и Боби поставят на първото блюдо на везната. 
            Трябва да задоволяват $0 \leq a_i < N$ за всяко $0\leq i \leq M-1$ и
            $a_i \neq a_j$ за всеки $0\leq i < j \leq M-1$.
        \item \code{b}: индексите $b_0, b_1, ..., b_{M-1}$, на монетите, които
            поставят на второто блюдо на везната. 
            Трябва да задоволяват $0 \leq b_i < N$ за всяко $0\leq i \leq M-1$ и
            $b_i \neq b_j$ за всеки $0\leq i < j \leq M-1$.
        \item Всяка монета може да се съдържа най-много в едно от блюдата, т.е.
            $a_i \neq a_j$ за всеки $0\leq i, j \leq M-1$
        \item Двете блюда трябва да съдържат равен брой монети, т.е. $a$ и $b$
            трябва да са с равна дължина.
        \item Нека монетите в първото блюдо тежат общо $A$, а тези във второто
            $B$.
            Функцията връща:
        \begin{itemize}
            \item $-1$, ако $A < B$
            \item $0$, ако $A = B$
            \item $1$, ако $A > B$
        \end{itemize}
        \item Функцията може да се вика $Q$ пъти за един тест.
    \end{itemize}
\end{itemize}

В случай, че наруши ограниченията за параметри на \code{compare\_coins}, ще
получите съобщение „Invalid question asked.“.

В случай, че извикате функцията \code{compare\_coins} повече от $Q$ пъти, ще
получите съощение „Too many questions asked.“.

\section*{Пример}
Грейдърът извършва следното извикване на функция:
\noskip
\begin{itemize}
    \item \code{find\_heavier(4)} Ави и Боби имат $N=4$ монети.
\end{itemize}
Програмата извършва следните извиквания на функции:
\noskip
\begin{itemize}
    \item \code{compare\_coins([0], [1])} връща -1
    \item \code{compare\_coins([2, 3], [1, 0])} връща 1
    \item \code{compare\_coins([1], [2])} връща 0 
    \item \code{compare\_coins([1], [3])} връща -1
\end{itemize}

По-леката монета е с индекс $0$, а по-тежката с индекс $3$.
Съответно \code{compare\_coins} трябва да върне $3$.

\section*{Подзадачи}
Във всички подзадачи $N = 2^k$.

В някои подзадачи, грейдърът \textit{не e адаптивен}, т.е. за всеки тест
индексите на по-леката и по-тежката монета са фиксирани преди изпълнението му.

В останлите подзадачи грейърът \textit{е адаптивен}, т.е. индексите на по-леката
и по-тежката монета могат да се определят в зависимост от въпросите, които
Вашата програма задава.
Разбира се, индексите са избират така, че да не противоречат на отговорите,
които грейдърът е дал на Вашата програма.

\begin{enumerate}
    \item (20 точки) $1 \leq N \leq 128$, $Q = 128$, гредърът
        \textit{не е адаптивен}
    \item (30 точки) $1 \leq N \leq 1024$, $Q = 30$, гредърът
        \textit{не е адаптивен}
    \item (40 точки) $1 \leq N \leq 1024$, $Q = 20$, гредърът
        \textit{е адаптивен}
    \item (10 точки) $1 \leq N \leq 1024$, $Q = 16$, гредърът
        \textit{е адаптивен}
\end{enumerate}
\section*{Локално тестване}
За да можете да тествате решението си на компютъра си, Ви се предоставят
файловете \code{Lgrader.cpp} и \code{fakecoins.h}, които да компилирате
заедно с Вашето решение \code{fakecoins.cpp}.

\noskip
\subsection*{Вход}
\noskip
\begin{itemize}
    \item Ред 1: две цели числа $N$ и $Q$
    \item Ред 2: две цели числа $a$ и $b$: индеските съответно на по-леката и
        по-тежката монета.
\end{itemize}
\noskip
\subsection*{Изход}
\noskip
\begin{itemize}
    \item Ред 1:
    \begin{itemize}
        \item \textit{„Output is correct.“},
            ако програмата е преминала успешно теста.
        \item \textit{„Invalid question.“},
            ако пограмата е задала въпрос, неотговарящ на гореописаните
            ограничения.
        \item \textit{„Too many questions asked.“},
            ако програмата е извършила повече от $Q$ извиквания на
            \code{compare\_coins}.
        \item \textit{„Output isn't correct.“},
            ако програмата не е надвишила максималния разрешен брой въпроси, но
            не е намерила правилно индекса на по-тежката монета.
    \end{itemize}
\end{itemize}

\section*{Изпращане на тестове към системата}
Можете да изпращате собствени тестове към системата. 

\noskip
\subsection*{Вход}
\noskip
\begin{itemize}
    \item Ред 1: две цели числа $N$ и $Q$
    \item Ред 2: 
    \begin{itemize}
        \item $0$ $a$ $b$, ако грейдърът не е адаптивен.
            $a$ и $b$ са съответно индеските на по-леката и по-тежката монета.
        \item $1$, ако грейдърът е адаптивен.
    \end{itemize}
\end{itemize}
\noskip
\subsection*{Изход}
\noskip
Форматът на изхода е същият като на локалния грейдър.
\end{document}
\makeatother
